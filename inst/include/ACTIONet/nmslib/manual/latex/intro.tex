\subsection{Objectives and History}

Non-Metric Space Library (NMSLIB) is an \textbf{efficient} and \textbf{extendable} cross-platform similarity search library and a toolkit for evaluation of similarity search methods.  The core-library does \textbf{not} have any third-party dependencies.

The goal of the project is to create an effective and \textbf{comprehensive} toolkit for searching in \textbf{generic metric and non-metric} spaces.
Even though the library contains a variety of metric-space access methods,
our main focus is on \emph{generic} and \emph{approximate} search methods,
in particular, on methods for non-metric spaces.
NMSLIB is possibly the first library with a principled support for non-metric space searching.

NMSLIB is an extendible library, which means that is possible to add new search methods and distance functions. NMSLIB can be used directly in C++ and Python (via Python bindings). In addition, it is also possible to build a query server, which can be used from Java (or other languages supported by Apache Thrift). Java has a native client, i.e., it works on many platforms without requiring a C++ library to be installed.

NMSLIB started as a personal project of Bilegsaikhan Naidan, who created the initial code base, the Python bindings,
and participated in earlier evaluations. 
The most successful class of methods--neighborhood/proximity graphs--is represented by the Hierarchical Navigable Small World Graph (HNSW)
due to Malkov and Yashunin \cite{Malkov2016}. 
Other most useful methods, include a modification of the Vantage-Point Tree (VP-tree) \cite{Boytsov_and_Bilegsaikhan:nips2013},
a Neighborhood APProximation index (NAPP) \cite{tellez2013succinct},
which was improved by David Novak,
as well as a vanilla uncompressed inverted file.

The current version of the manual focuses solely on the description of:
\begin{itemize}
    \item search spaces and distances (see \S~\ref{SectionSpaces});
    \item search methods (see \S~\ref{SectionMethods}).
\end{itemize}

Details about building/installing, benchmarking, extending the code, as well as basic
tuning guidelines, can be found online: \url{https://github.com/nmslib/nmslib/tree/v\LibVersion/manual/README.md}.

\section{Terminology and Problem Formulation}\label{SectionProbForm}
Similarity search is an essential part of many applications,
which include, among others,  
content-based retrieval of multimedia  and statistical machine learning.
The search is carried out in a finite database of objects $\{o_i\}$,
using a search query $q$ and a dissimilarity measure (the term data point or simply a point is often
used a synonym to denote either a data object or a query).
 The dissimilarity measure is typically represented by a distance function $d(o_i, q)$. 
The ultimate goal is to answer a query by retrieving a subset of database objects sufficiently similar to the query $q$.
These objects will be called \emph{answers}.
A combination of data points and the distance function is called a \emph{search space},
or simply a \emph{space}.


Note that we use the terms \emph{distance} and the \emph{distance function} in a broader sense than
some of the textbooks:
We do not assume that the distance is a true metric distance. 
The distance function can disobey the triangle inequality and/or be even non-symmetric.

Two retrieval tasks are typically considered: a nearest neighbor and a range search. 
The nearest neighbor search aims to find the least dissimilar object,
i.e., the object at the smallest distance from the query.
Its direct generalization is the $k$-nearest neighbor search (the \knn search),
which looks for the $k$  closest objects.
Given a radius $r$, 
the range query retrieves all objects within a query ball (centered at the query object $q$) with the radius $r$,
or, formally, all the objects~$\lbrace o_i \rbrace$ such that $d(o_i, q) \le r$. 
In generic spaces, the distance is not necessarily symmetric. 
Thus, two types of queries can be considered. 
In a  \emph{left} query, the object is the left argument of the distance function,
while the query is the right argument.
In a \emph{right} query, $q$ is the first argument and the object is the second, i.e.,
the right, argument.

The queries can be answered either exactly, 
i.e., by returning a complete result set that does not contain erroneous elements, or, 
approximately, e.g., by finding only some answers.
Thus, the methods are evaluated in terms of efficiency-effectiveness trade-offs
rather than merely in terms of their efficiency.
One common effectiveness metric is recall. In the case
of the nearest neighbor search, it is computed as
an average fraction of true neighbors returned by the method with ties broken arbitrarily.
