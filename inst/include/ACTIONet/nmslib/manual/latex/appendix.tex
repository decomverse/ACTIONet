\begin{appendix}
\section{Description of Projection Types}\label{SectionProjDetails}
\subsection{The classic random projections} 
The classic random projections work only for vector spaces (both sparse and dense).
At index time, we generate \ttt{projDim} vectors by sampling their elements from
the standard normal distribution $\mathcal{N}(0,1)$ and orthonormalizing them. \footnote{If 
the dimensionality of the projection space is larger than the dimensionality of the original
space, only the first \ttt{projDim} vectors are orthonormalized. The remaining are simply
divided by their norms.}
Coordinates in the projection spaces are obtained by computing scalar products
between a given vector and each of the \ttt{projDim} randomly generated vectors.

In the case of sparse vector spaces, the dimensionality is first reduced via the hashing trick:
the value of the element $i$ is equal to the sum of values for all elements whose indices are 
hashed into number $i$. 
After hashing, classic random projections are applied. 
The dimensionality of the intermediate space is defined by a method's parameter \ttt{intermDim}. 

The hashing trick is used purely for efficiency reasons. 
However, for large enough values of the intermediate
dimensionality, it has virtually no adverse affect on performance.
For example, in the case of Wikipedia tf-idf vectors (see  
\url{https://github.com/nmslib/nmslib/tree/v\LibVersion/manual/datasets.md}),
it is safe to use the value \ttt{intermDim}=4096.

Random projections work best if both the source and the target space are Euclidean,
whereas the distance is either $L_2$ or the cosine distance.
In this case, there are theoretical guarantees that the projection preserves
well distances in the original space (see e.g. \cite{bingham2001random}).

\subsection{FastMap} 
FastMap introduced by Faloutsos and Lin \cite{faloutsos1995fastmap}
is also a type of the random-projection method. 
At indexing time, we randomly select $projDim$ pairs $A_i$ and $B_i$.
The \mbox{$i$-\textit{th}} coordinate of vector $x$ is computed using the formula:
\begin{equation}\label{EqFastMap}
\mbox{\ttt{FastMap}}_{i}(x)  = \frac{d(A_i,x)^2 - d(B_i,x_i)^2 + d(A_i,B_i)}{2 d(A_i,B_i)^2}
\end{equation}
Given points $A$ and $B$ in the Euclidean space, Eq.~\ref{EqFastMap} gives the length of the
orthogonal projection of $x$ to the line connecting $A$ and $B$.
However, FastMap can be used in non-Euclidean spaces as well.

\subsection{Distances to the Random Reference Points} 
This method is a folklore projection approach,
where the \mbox{$i$-\textit{th}} coordinate of point $x$ in the projected space is computed as simply $d(x, \pi_i)$,
where $\pi_i$ is a pivot in the original space, i.e., a randomly selected reference point.
Pivots are selected once during indexing time.

\subsection{Permutation-based Projections.}
In this approach, we also select \ttt{projDim} pivots at index time.
However, instead of using raw distances to the pivots,
we rely on ordinal positions of pivots sorted by their distance to a point.  
A more detailed description is given in \S~\ref{SectionPermMethod}.
\end{appendix}
